\chapter*{Agradecimentos}
\label{CHP:ACKNOWLEDGMENT}%%
\thispagestyle{empty}
% \PARstartOne{A}{grade\c{c}o}
Quero expressar meus sinceros agradecimentos ao Professor Paulo C\'{e}sar Cortez pela sua inestim\'{a}vel orienta\c{c}\~{a}o, amizade, exemplo de dedica\c{c}\~{a}o \`{a} ci\^{e}ncia e que com sua experi\^{e}ncia e paci\^{e}ncia tornou poss\'{\i}vel a realiza\c{c}\~{a}o deste trabalho.

Ao Professor Jo\~{a}o C\'{e}sar Moura Mota, por seu constante incentivo, amizade, orienta\c{c}\~{a}o e que sempre apostou no meu desenvolvimento cient\'{\i}fico.

Ao Professor Carlos Jacinto de Oliveira que n\~{a}o economizou esfor\c{c}os no apoio ao desenvolvimento deste trabalho principalmente nas horas mais cr\'{\i}ticas, em especial no meu afastamento das atividades docentes na Universidade Estadual do Cear\'{a} bem como em meu est\'{a}gio no Max-Planck-Institute for Chemistry da Universidade de Mainz-Alemanha.

Ao Professor Stephan Borrmann, diretor do Departamento de Qu\'{\i}mica de Part\'{\i}culas do Max-Planck-Institute for Chemistry, por sua amizade e apoio decisivo para a realiza\c{c}\~{a}o dos experimentos.

N\~{a}o posso deixar de agradecer aos meus novos amigos Johannes Schneider, Julia Schmale, Paul Reitz, Friederike Freutel, Miklos Szakall, Frank Drewnick, St\'{e}phane Gallavardin, Johannes Fachinger, Katja Dzepina,  Karin Sulsky, Rosemarie Gross, pelo valioso apoio t\'{e}cnico, e que com amizade tornaram menos dif\'{\i}ceis os quatro meses que passei longe de minha casa.

Ao Professor Humberto Carmona, por sua amizade, incentivo e valiosas cr\'{\i}ticas e sugest\~{o}es na composi\c{c}\~{a}o deste trabalho.

Aos companheiros de doutorado do Laborat\'{o}rio de Teleinform\'{a}tica, John Hebert da Silva Felix, Auzuir Ripardo Alexandria  e Rodrigo Carvalho Sousa Costa, pelas sugest\~{o}es e ideias ao longo do desenvolvimento deste trabalho.

Ao Professor Gerson Paiva Almeida pelo apoio e sugest\~{o}es.

Ao amigo Manuel Pereira pelo incentivo, pelas suas ideias que demonstram toda a sua criatividade.

Aos professores do Colegiado do Curso de F\'{\i}sica da Universidade Estadual do Cear\'{a} que, com sacrif\'{\i}cio pr\'{o}prio, apoiaram este trabalho substituindo-me nas minhas tarefas docentes.

Ao Max-Planck-Institute for Chemistry pela oportunidade de desenvolver parte essencial do meu trabalho em seus laborat\'{o}rios.

\`{A} Funda\c{c}\~{a}o Cearense de Meteorologia e Recursos H\'{\i}dricos - FUNCEME e \`{a} Financiadora de Estudos e Projetos - FINEP pelo apoio financeiro no desenvolvimento do prot\'{o}tipo do CCNC-SDCC, objeto deste trabalho.

\`{A} Funda\c{c}\~{a}o Cearense de Apoio ao Desenvolvimento Cient\'{\i}fico e Tecnol\'{o}gico - FUNCAP pela bolsa concedida que possibilitou minha temporada no Max-Planck-Institute for Chemistry.

Aos meus pais, Geraldo Pinheiro e Filomena, que n\~{a}o mediram esfor\c{c}os na educa\c{c}\~{a}o de seus filhos.

\`{A} minha av\'{o}, Geralda (em mem\'{o}ria), pelo carinho e amor a mim dedicados. \\


%/
%/


A Deus por tudo. 