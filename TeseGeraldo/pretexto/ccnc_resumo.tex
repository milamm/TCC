\chapter*{Resumo}
\label{CHP:RESUM0}%%
\thispagestyle{empty}




\PARstartOne{E}{sta} Tese prop\~{o}e um novo contador est\'{a}tico de n\'{u}cleos de condensa\c{c}\~{a}o de nuvens baseado em t\'{e}cnicas de vis\~{a}o computacional. O processo concebido nesse novo contador consiste na captura de uma amostra do ar atmosf\'{e}rico dentro de uma c\^{a}mara de nuvens est\'{a}tica supersaturada de vapor de \'{a}gua, produzindo got\'{\i}culas de \'{a}gua. Essas got\'{\i}culas, ao ca\'{\i}rem por gravidade, cruzam um feixe de LASER que define um volume de amostragem, tornando-as vis\'{\i}veis. Uma s\'{e}rie de imagens deste processo \'{e} digitalizada e processada para permitir a contagem das got\'{\i}culas presentes nesse volume de amostragem. Tais got\'{\i}culas s\~{a}o automaticamente contadas por um sistema de vis\~{a}o computacional composto por t\'{e}cnicas de binariza\c{c}\~{a}o por limiar, transformada de dist\^{a}ncia e transformada watershed. Esse volume de amostragem \'{e} calculado atrav\'{e}s de uma nova metodologia proposta nesta tese. Essa metodologia torna desnecess\'{a}ria a realiza\c{c}\~{a}o de complexos procedimentos de calibra\c{c}\~{a}o do contador desenvolvido bem como de outros similares. Um prot\'{o}tipo foi montado e experimentos baseados na compara\c{c}\~{a}o entre instrumentos foram realizados. Os resultados indicam que os procedimentos aplicados s\~{a}o adequados na determina\c{c}\~{a}o da concentra\c{c}\~{a}o dos n\'{u}cleos de condensa\c{c}\~{a}o de nuvens e que o equipamento desenvolvido pode ser usado em altas concentra\c{c}\~{o}es, situa\c{c}\~{a}o em que outros equipamentos equivalentes n\~{a}o s\~{a}o confi\'{a}veis, devido ao efeito da sobreposi\c{c}\~{a}o de got\'{\i}culas nas imagens analisadas. Al\'{e}m disso, a resolu\c{c}\~{a}o temporal foi significativamente melhorada e a intensa utiliza\c{c}\~{a}o da eletr\^{o}nica digital tamb\'{e}m permitiu a redu\c{c}\~{a}o de volume, de peso e de consumo de energia deste prot\'{o}tipo.



%\newline
%\newline
\noindent \textbf{Palavras-chaves:}  transformada \emph{watershed}; transformada de dist\^{a}ncia; processamento de imagens digitais; aerossol atmosf\'{e}rico; instrumenta\c{c}\~{a}o meteorol\'{o}gica.
