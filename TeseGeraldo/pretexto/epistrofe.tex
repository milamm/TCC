%\textbf{DEDICAT\'{O}RIA}
\thispagestyle{empty}%
\newpage
\null\vfill
\begin{flushright}

"Ainda que eu falasse as l\'{\i}nguas dos homens e dos anjos, e n\~{a}o tivesse amor, seria como o metal que soa ou como o sino que tine.

E ainda que tivesse o dom de profecia, e conhecesse todos os mist\'{e}rios e toda a ci\^{e}ncia, e ainda que tivesse toda a f\'{e}, de maneira tal que transportasse os montes, e n\~{a}o tivesse amor, nada seria.

E ainda que distribu\'{\i}sse toda a minha fortuna para sustento dos pobres, e ainda que entregasse o meu corpo para ser queimado, e n\~{a}o tivesse amor, nada disso me aproveitaria.

O amor \'{e} sofredor, \'{e} benigno; o amor n\~{a}o \'{e} invejoso; o amor n\~{a}o trata com leviandade, n\~{a}o se ensoberbece.

N\~{a}o se porta com indec\^{e}ncia, n\~{a}o busca os seus interesses, n\~{a}o se irrita, n\~{a}o suspeita mal;

N\~{a}o folga com a injusti\c{c}a, mas folga com a verdade;

Tudo sofre, tudo cr\^{e}, tudo espera, tudo suporta.

O amor nunca falha; mas havendo profecias, ser\~{a}o aniquiladas; havendo l\'{\i}nguas, cessar\~{a}o; havendo ci\^{e}ncia, desaparecer\'{a};

Porque, em parte, conhecemos, e em parte profetizamos;

Mas, quando vier o que \'{e} perfeito, ent\~{a}o o que o \'{e} em parte ser\'{a} aniquilado.

Quando eu era menino, falava como menino, sentia como menino, discorria como menino, mas, logo que cheguei a ser homem, acabei com as coisas de menino.

Porque agora vemos por espelho em enigma, mas ent\~{a}o veremos face a face; agora conhe\c{c}o em parte, mas ent\~{a}o conhecerei como tamb\'{e}m sou conhecido.

Agora, pois, permanecem a f\'{e}, a esperan\c{c}a e o amor, estes tr\^{e}s, mas o maior destes \'{e} o amor."





\emph{Cor\'{\i}ntios I, cap. 13; vers. 1 a 13}
.
\end{flushright}
   \vspace{3.0cm}
