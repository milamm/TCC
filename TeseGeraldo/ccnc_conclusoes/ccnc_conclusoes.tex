\doublespacing
%% ------------------------------------------------------------------------- %%

\chapter{Conclus\~{o}es, contribui\c{c}\~{o}es e trabalhos futuros}
\label{cap:conclus\~{o}es}

\PARstartOne{E}{sta} tese prop\~{o}e um novo contador de n\'{u}cleos de condensa\c{c}\~{a}o de nuvens com c\^{a}mara de difus\~{a}o est\'{a}tica, empregando um sistema de vis\~{a}o computacional para a contagem autom\'{a}tica de gotas. Al\'{e}m disso, s\~{a}o descritos e analisados os diversos tipos de CCNCs existentes, apontando-se as principais caracter\'{\i}sticas, bem como suas limita\c{c}\~{o}es. Descreve-se a metodologia de concep\c{c}\~{a}o e desenvolvimento desse novo CCNC, dividindo-se em duas partes: hardware e software.

Quanto \`{a} primeira parte, contempla o esquem\'{a}tico do novo CCNC, o controle de supersatura\c{c}\~{a}o, composto por circuitos el\'{e}tricos projetados especificamente para este fim, contendo um reduzido n\'{u}mero de componentes. A segunda parte, o software, divide-se em dois n\'{u}cleos, sendo o primeiro respons\'{a}vel por permitir a intera\c{c}\~{a}o do novo CCNC com o operador para possibilit\'{a}-lo operar os servi\c{c}os e a configura\c{c}\~{a}o operacional, tais como ajustar o volume de amostragem, o n\'{\i}vel de ilumina\c{c}\~{a}o, entre outros.  O segundo n\'{u}cleo \'{e} composto pelo sistema de vis\~{a}o computacional, composto por um conjunto de algoritmos, que permite a segmenta\c{c}\~{a}o e contagem autom\'{a}tica das gotas, demonstrando-se eficiente mesmo em altas concentra\c{c}\~{o}es em que ocorrem sobreposi\c{c}\~{a}o de gotas. Em equipamentos equivalentes, essa situa\c{c}\~{a}o \'{e} contornada apenas com a utiliza\c{c}\~{a}o de diluidores de gases. Al\'{e}m disso, esse sistema de vis\~{a}o \'{e} respons\'{a}vel pela determina\c{c}\~{a}o autom\'{a}tica do volume de amostragem da c\^{a}mara de nuvens.  A valida\c{c}\~{a}o do novo CCNC \'{e} obtida por compara\c{c}\~{a}o com um CPC, produzindo-se concentra\c{c}\~{o}es de aeross\'{o}is neste e comparando-se com as concentra\c{c}\~{o}es medidas pelo novo CCNC, resultando numa correla\c{c}\~{a}o de 99\% entre as medidas.

Com base nos resultados experimentais conclui-se que o novo CCNC pode ser empregado na medi\c{c}\~{a}o de CCNs, inclusive em situa\c{c}\~{o}es de atmosferas polu\'{\i}das, onde h\'{a} alta concentra\c{c}\~{a}o de aeross\'{o}is. Conclui-se tamb\'{e}m que esse CCNC, por possuir baixos peso e consumo de energia e pode ser embarcado mais facilmente em aeronaves menores.
	
A expans\~{a}o da capacidade de contagem de gotas para cerca de 4000 part\'{\i}culas/cm$^3$, sem o emprego de diluidores de gases e bem acima de outros CCNCs como, por exemplo, o CCNC-MPIChemie (CCNC do Instituto Max Plank para Qu\'{\i}mica da Universidade de Mainz-Alemanha), constitui uma contribui\c{c}\~{a}o importante, pois, permite a medi\c{c}\~{a}o de CCNs em muitas situa\c{c}\~{o}es que atualmente s\~{a}o de interesse cient\'{\i}fico como no caso de atmosferas polu\'{\i}das.
	
Uma nova metodologia para determina\c{c}\~{a}o do valor do volume de amostragem tamb\'{e}m \'{e} apresentada, sendo que essa garante medidas confi\'{a}veis dentro de uma ampla faixa de concentra\c{c}\~{a}o de n\'{u}cleos  de condensa\c{c}\~{a}o de nuvens e dispensa necessidade de uma calibra\c{c}\~{a}o em bancada.
	
Al\'{e}m das contribui\c{c}\~{o}es mencionadas, destacam-se tamb\'{e}m que todo o material usado no projeto do novo CCNC \'{e} facilmente encontrado no mercado nacional e a intensiva utiliza\c{c}\~{a}o de tecnologia digital permitiu a constru\c{c}\~{a}o de um prot\'{o}tipo de baixo peso, de baixo consumo de energia e de pequeno volume se comparado com outros contadores equivalentes.

Como sugest\~{a}o de trabalhos futuros, visando o aperfei\c{c}oamento do novo CCNC, podem-se relacionar:

1.	substituir o a fonte de luz LASER a g\'{a}s (HeNe) por outra fonte de luz LASER baseada em semicondutor. Pode-se citar tr\^{e}s bons motivos: redu\c{c}\~{a}o de peso, de consumo de energia e de custo do CCNC-SDCC. Do ponto de vista mec\^{a}nico este \'{e} um procedimento simples porque a c\^{a}mara de nuvens foi constru\'{\i}da com acess\'{o}rios prevendo essa possibilidade. Entretanto no que diz respeito a metodologia de determina\c{c}\~{a}o do volume de amostragem, essa dever\'{a} sofrer adapta\c{c}\~{o}es pois normalmente a geometria do feixe do LASER semicondutor n\~{a}o \'{e} perfeitamente cil\'{\i}ndrico;

2.	embarcar o sistema de vis\~{a}o computacional e de interface homem m\'{a}quina no \emph{hardware} do pr\'{o}prio CCNC-SDCC. Isto feito, dispensa a utiliza\c{c}\~{a}o de um computador para tal. Nesse sentido, passos j\'{a} foram dados, pois, o c\'{o}digo de processamento de imagem foi implementado em linguagem de program\c{c}\~{a}o C-ANSI;

3.	colocar outra c\^{a}mera digital numa janela da c\^{a}mara para permitir uma vis\~{a}o tridimensional das gotas, podendo ser rastreado o processo de crescimento, inclusive, extraindo-se medidas;

4.	dispor o CCNC de um sistema elet\^{o}nico que permita a transmiss\~{a}o dos dados por GPRS, facilitando o embarque deste equipamento em microaeronaves n\~{a}o tripuladas;

5. disponibilizar o histograma de tamanho de got\'{\i}culas. Isto \'{e} totalmente fact\'{\i}vel a partir da substitui\c{c}\~{a}o da atual \emph{webcam} por outra de alta defini\c{c}\~{a}o de modo a permitir uma melhor rela\c{c}\~{a}o  \emph{pixels}/milimetros.
