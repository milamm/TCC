\pdfbookmark[1]{Resumo}{CHP:RESUM0}
\chapter*{Resumo}
\label{CHP:RESUM0}
\thispagestyle{empty}
\acresetall
\PARstartOne{O}{s} sistemas de diagn�stico de falhas v�m adquirindo import�ncia na computa��o devido � complexidade dos dispositivos digitais. Seu uso na identifica��o de problemas de \emph{hardware} tem se tornado uma demanda crescente tanto para empresas especializadas em computadores, como montadoras e fabricantes e mesmo para usu�rios dom�sticos, que desejam verificar a integridade do seu equipamento.

Neste trabalho, foi desenvolvida uma ferramenta de diagn�stico de falhas em mem�rias, chamada MDiag. O \emph{software} foi constru�do como uma aplica��o para sistemas operacionais Linux. Todo o projeto e implementa��o foi embasado por um amplo estudo sobre falhas em mem�rias, algoritmos de detec��o de falhas em mem�rias e o controle deste componente atrav�s do Linux.

Tamb�m foi desenvolvido um sistema autom�tico de gera��o e inser��o de falhas, utilizando funcionalidades de \emph{debug} presentes na maioria dos processadores atuais. Este sistema foi usado para testar e validar o MDiag quanto a cobertura de falhas.

Foram realizados testes reais com placas de mem�rias defeituosas, onde os resultados do MDiag foram comparados com os de ferramentas utilizadas no mercado.
\newline

\noindent \textbf{Palavras-chaves:} Modelagem, Teste, Algoritmo, Cobertura, Complexidade, LinuxMM.
