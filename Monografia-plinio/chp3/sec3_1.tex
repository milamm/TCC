\section{Ambiente de Desenvolvimento e Teste}

Para o desenvolvimento deste trabalho, foram utilizados diversos tipos de computadores a fim de garantir compatibilidade com a maior variedade poss�vel de plataformas. Sistemas variando desde kit de desenvolvimento de sistema embarcado at� servidores de grande porte com distribui��es Linux variadas. Todas as m�quinas utilizadas, com exce��o de um computador pessoal do autor, fazem parte do acervo do \ac{LESC} da \ac{UFC}.

Nenhum tipo de bibliotecas, al�m das padr�es do Linux, foram necess�rias durante o desenvolvimento, pois, para favorecer a portabilidade, a ferramenta utiliza apenas funcionalidades presentes no pr�prio \emph{kernel} do Linux, a partir da vers�o 2.6.9.

Para simular as falhas em mem�ria, foi utilizada uma abordagem baseada em \cite{PETRU:2002} (descrita com detalhes na se��o \ref{SEC:SIM}). Um \emph{software} de \emph{debug} com capacidade de estabelecer \emph{breakpoints} em n�vel de \emph{hardware} foi utilizado para interromper a execu��o do teste no momento desejado e escrever um valor de erro na mem�ria, simulando qualquer tipo de falha. O \emph{software} utilizado foi o \ac{GDB}, uma das mais consagradas ferramentas de \emph{debug} para Linux.

Foram realizados testes com mem�rias defeituosas reais e os resultados foram comparados aqueles obtidos por outras ferramentas de diagn�stico do mercado. Os componentes e o procedimento desta avalia��o s�o detalhados na Se��o \ref{SEC:REAL} 