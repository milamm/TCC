\chapter{Metodologia}
\label{CHP:MET}

Neste cap�tulo s�o descritos os ambientes utilizados no desenvolvimento, avalia��o e teste da ferramenta proposta, chamada de MDiag. Algumas caracter�sticas importantes da implementa��o, que garantem a efic�cia do diagn�stico, s�o apresentadas na Se��o \ref{SEC:DES}. Um sistema autom�tico de inser��o de falhas desenvolvido para valida��o do MDiag atrav�s de simula��o, � detalhado na Se��o \ref{SEC:SIM}. Por fim, � descrito o procedimento de testes em ambientes reais.

    %%% Se��o 3.1:
    \section{Ambiente de Desenvolvimento e Teste}

Para o desenvolvimento deste trabalho, foram utilizados diversos tipos de computadores a fim de garantir compatibilidade com a maior variedade poss�vel de plataformas. Sistemas variando desde kit de desenvolvimento de sistema embarcado at� servidores de grande porte com distribui��es Linux variadas. Todas as m�quinas utilizadas, com exce��o de um computador pessoal do autor, fazem parte do acervo do \ac{LESC} da \ac{UFC}.

Nenhum tipo de bibliotecas, al�m das padr�es do Linux, foram necess�rias durante o desenvolvimento, pois, para favorecer a portabilidade, a ferramenta utiliza apenas funcionalidades presentes no pr�prio \emph{kernel} do Linux, a partir da vers�o 2.6.9.

Para simular as falhas em mem�ria, foi utilizada uma abordagem baseada em \cite{PETRU:2002} (descrita com detalhes na se��o \ref{SEC:SIM}). Um \emph{software} de \emph{debug} com capacidade de estabelecer \emph{breakpoints} em n�vel de \emph{hardware} foi utilizado para interromper a execu��o do teste no momento desejado e escrever um valor de erro na mem�ria, simulando qualquer tipo de falha. O \emph{software} utilizado foi o \ac{GDB}, uma das mais consagradas ferramentas de \emph{debug} para Linux.

Foram realizados testes com mem�rias defeituosas reais e os resultados foram comparados aqueles obtidos por outras ferramentas de diagn�stico do mercado. Os componentes e o procedimento desta avalia��o s�o detalhados na Se��o \ref{SEC:REAL} 
    %%% Se��o 3.2:
    \input{chp3/sec3_2}
    %%% Se��o 3.3:
    \input{chp3/sec3_3}
    %%% Se��o 3.4:
    \section{Teste em Ambientes Reais}
\label{SEC:REAL}

Al�m do ambiente de simula��o de falhas descrito na se��o anterior, o MDiag tamb�m foi submetido a situa��es de uso reais a fim de garantir sua utilidade pr�tica.

O acervo utilizado para teste foi composto por dez placas de mem�ria, algumas em perfeito funcionamento, outras com falhas. Metade delas possu�am encapsulamento \ac{SO-DIMM}, pr�prias para computadores de dimens�es reduzidas, como \emph{notebooks} e \emph{netbooks}, e as outras, encapsulamento \ac{DIMM}, geralmente usadas nos computadores pessoais comuns, estilo \emph{desktop}, ou em servidores.

Nesses testes, o MDiag confrontou dois \emph{softwares} de diagn�stico consagrados no mercado. O primeiro foi o \ac{LTT} \cite{LTT:2011}, desenvolvido pela PC-Doctor \cite{PCDOCTOR:2011} para os computadores da fabricante Lenovo. Na realidade este produto re�ne um conjunto de diagn�sticos que cobre quase todos os componentes da m�quina. O segundo foi o Memtest86+ \cite{MEMTEST:2011}, umas das ferramentas de diagn�stico de mem�ria mais abrangentes em termos de cobertura de falhas.

� importante ressaltar que o \ac{LTT} foi utilizado com o sistema operacional Windows, enquanto o Memtest86+ � uma ferramenta \emph{stand-alone} que executa diretamente de uma m�dia externa, como um CD ou \emph{pendrive} sem carregar nenhum \ac{SO}. Estas caracter�sticas influenciaram bastante nos resultados, pois afetam diretamente a quantidade de mem�ria testada.

O teste com cada ferramenta foi executado cinco vezes para cada m�dulo de mem�ria. Estas, por sua vez, foram etiquetadas cegamente, n�o havendo nenhum conhecimento pr�vio sobre a presen�a ou aus�ncia de falhas em cada uma delas.

Uma estimativa do tempo de execu��o de cada algoritmo implementado pelo MDiag foi elaborada com base na sua complexidade e levando em considera��o o \emph{overhead} das opera��es realizadas entre as escritas e leituras. Um algoritmo pode exigir apenas $10 N$ opera��es de acesso a mem�ria, mas para ser realizado ainda � necess�rio alocar mem�ria, executar checagens ap�s as leituras, incrementar o contador de endere�o, desalocar mem�ria, etc. Assim, o total de opera��es de uma implementa��o deve ser maior que o simples valor da complexidade.

A estimativa utilizou a f�rmula \ref{EQU:FUNCLOGIS}, na qual $C$ � a complexidade do algortimo e $O$ � um par�metro de tempo m�dio por opera��o, levando-se em conta alguns processamentos extras necess�rios. O valor de $O$ foi medido para cada algoritmo, individualmente, dividindo-se o tempo gasto para executar um elemento de teste pela quantidade de opera��es naquele elemento.

\begin{equation}
{T_{est} = C \cdot O}
\label{EQU:FUNCLOGIS}
\end{equation}


\section{Resumo do Cap�tulo}

Neste cap�tulo foram descritos os principais passos e medidas tomadas no desenvolvimento da aplica��o.
Tamb�m foi abordada a metodologia de valida��o elaborada para medir a efici�ncia dos algoritmos escolhidos.

No cap�tulo seguinte, s�o apresentados os resultados dos testes de inser��o de falhas utilizados para validar a ferramenta, al�m de medir o desempenho desta em rela��o a ferramentas do mercado.

