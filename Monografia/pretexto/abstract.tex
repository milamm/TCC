\chapter*{Abstract}
\label{CHP:ABSTRACT}%%
\thispagestyle{empty}

%\PARstartOne{O}{s} sistemas de diagn�stico de falhas v�m adiquirindo import�ncia na computa��o devido � complexidade dos dispositivos digitais. Seu uso na identifica��o de problemas de \emph{hardware} tem se tornado uma demanda crescente tanto para empresas especializadas em computadores, como montadoras e fabricantes, quanto para usu�rios dom�sticos que desejam verificar a integridade do seu equipamento.

\PARstartOne{F}{ault} diagnosis systems has growing in importance on the computing due to the complexity of the digital devices. Its use to identiying hardware's malfunction has become an ascenceding demand for both, companies specialized in computers, such as assemblers and manufacturers, as well as for home users who want to check the integrity of your equipment.

%Neste tabalho, foi desenvolvida uma ferramenta de diagn�stico de falhas em mem�rias, chamada MDiag. O \emph{software} foi constru�do como uma aplica��o para sistemas operacionais Linux. Todo o projeto e implementa��o foi embasado por um amplo estudo sobre falhas em mem�rias, algitmos de detec��o de falhas em mem�rias e o controle deste componente atrav�s do Linux.

In this study, we developed a tool for fault diagnosis in memories, called MDiag. It was built as an application for Linux operating systems. All the design and implementation was based on an extensive study of memory faults, fault detection algorithms and Linux memory menagement.

%Tamb�m foi desenvolvido um sistema autom�tico de gera��o e inser��o de falhas utilizando funcionalidades de \emph{debug} presentes na maioria dos processadores atuais. Este sistema foi usado para testar e validar o MDiag quanto a cobertura de falhas.

It was also developed an automatic system for fault generation and insertion using debug features available in most of current processors. This system was used to test and validate the MDiag as its fault coverage.

%Foram realizados testes reais com placas de mem�rias defeituosas, onde os resultados do MDiag foram comparados com os de ferramentas utilizadas no mercado.

Tests were carried out with real faulty memories, where the results where the results achieved by MDiag were compared with those achieved by tools used in the market.
\newline

\noindent \textbf{Keywords:} Modeling, Testing, Algorithm, Coverage, Complexity, LinuxMM.
